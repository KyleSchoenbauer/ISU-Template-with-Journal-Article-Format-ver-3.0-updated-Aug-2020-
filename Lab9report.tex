\documentclass{article}

\usepackage{fancyhdr}
\usepackage{extramarks}
\usepackage{amsmath}
\usepackage{amsthm}
\usepackage{amsfonts}
\usepackage{tikz}
\usepackage{float}

\usetikzlibrary{automata,positioning}

%
% Basic Document Settings
%

\topmargin=-0.45in
\evensidemargin=0in
\oddsidemargin=0in
\textwidth=6.5in
\textheight=9.0in
\headsep=0.25in

\linespread{1.1}

\pagestyle{fancy}
\lhead{\hmwkAuthorName}
\chead{\hmwkClass\ (\hmwkClassInstructor): \hmwkTitle}
\rhead{\firstxmark}
\lfoot{\lastxmark}
\cfoot{\thepage}

\renewcommand\headrulewidth{0.4pt}
\renewcommand\footrulewidth{0.4pt}

\setlength\parindent{0pt}

%
% Create Problem Sections
%

\newcommand{\enterProblemHeader}[1]{
    \nobreak\extramarks{}{Problem \arabic{#1} continued on next page\ldots}\nobreak{}
    \nobreak\extramarks{Problem \arabic{#1} (continued)}{Problem \arabic{#1} continued on next page\ldots}\nobreak{}
    }
    \newcommand{\exitProblemHeader}[1]{
    \nobreak\extramarks{Problem \arabic{#1} (continued)}{Problem \arabic{#1} continued on next page\ldots}\nobreak{}
    \stepcounter{#1}
    \nobreak\extramarks{Problem \arabic{#1}}{}\nobreak{}
}

\setcounter{secnumdepth}{0}
\newcounter{partCounter}
\newcounter{homeworkProblemCounter}
\setcounter{homeworkProblemCounter}{1}
\nobreak\extramarks{Problem \arabic{homeworkProblemCounter}}{}\nobreak{}

%
% Homework Problem Environment
%
% This environment takes an optional argument. When given, it will adjust the
% problem counter. This is useful for when the problems given for your
% assignment aren't sequential. See the last 3 problems of this template for an
% example.
%
\newenvironment{homeworkProblem}[1][-1]{
    \ifnum#1>0
        \setcounter{homeworkProblemCounter}{#1}
    \fi
    \section{Problem \arabic{homeworkProblemCounter}}
    \setcounter{partCounter}{1}
    \enterProblemHeader{homeworkProblemCounter}
}{
    \exitProblemHeader{homeworkProblemCounter}
}

%
% BEGIN MODIFCATIONS HERE
%
% Homework Details
%   - Title
%   - Due date
%   - Class
%   - Section/Time
%   - Instructor
%   - Author
%
\newcommand{\hmwkTitle}{Lab 9}
\newcommand{\hmwkDueDate}{Oct. 20th}
\newcommand{\hmwkClass}{AER E 361}
\newcommand{\hmwkClassInstructor}{Professor Nelson}
\newcommand{\hmwkAuthorName}{\textbf{Kyle Schoenbauer}}
%
% END MODIFCATIONS HERE
%

%
% Title Page
%

\title{
    \vspace{2in}
    \textmd{\textbf{\hmwkClass:\ \hmwkTitle}}\\
    \normalsize\vspace{0.1in}\small{Due\ on\ \hmwkDueDate\ at 3:10pm}\\
    \vspace{0.1in}\large{\textit{\hmwkClassInstructor}}
    \vspace{3in}
}


\author{\hmwkAuthorName}
\date{}

\renewcommand{\part}[1]{\textbf{\large Part \Alph{partCounter}}\stepcounter{partCounter}\\}


 % Various Helper Commands
%

% Useful for algorithms
\newcommand{\alg}[1]{\textsc{\bfseries \footnotesize #1}}

% For derivatives
\newcommand{\deriv}[1]{\frac{\mathrm{d}}{\mathrm{d}x} (#1)}

% For partial derivatives
\newcommand{\pderiv}[2]{\frac{\partial}{\partial #1} (#2)}

% Integral dx
\newcommand{\dx}{\mathrm{d}x}

% Alias for the Solution section header
\newcommand{\solution}{\textbf{\large Solution}}

% Probability commands: Expectation, Variance, Covariance, Bias
\newcommand{\E}{\mathrm{E}}
\newcommand{\Var}{\mathrm{Var}}
\newcommand{\Cov}{\mathrm{Cov}}
\newcommand{\Bias}{\mathrm{Bias}}

\begin{document}

\maketitle

\pagebreak

\begin{homeworkProblem}



%Please add the following required packages to your document preamble:
% \usepackage{multirow}
\begin{table}[]
\caption{Signed and Unsigned Values}
\label{my-label}
\begin{tabular}{lllll}
\multirow{2}{*}{Size} & \multicolumn{2}{c}{Unsigned} & \multicolumn{2}{c}{Signed} \\
& Min. Value & Max. Value & Min. Value & Max. Value \\
8-bit &0 &255 &-128 &127 \\
16-bit &0 &65535 &-32768 &32767 \\
32-bit &0 &4294967295 &-2147483648 &2147483647 \\
64-bit &0 &18446744073709551616 &-9223372036854775808 &9223372036854775807
\end{tabular}
\end{table}
See table above.
\end{homeworkProblem}

\begin{homeworkProblem}
8-bit unsigned representation

88 = 01011000

0 = 00000000

1 = 00000001

127 = 01111111

255 = 11111111
\end{homeworkProblem}
\begin{homeworkProblem}
8-bit 2's complement signed representation

+88 = 01011000

-44 = 11010100

-1 = 11111111

0 = 00000000

+1 = 00000001

-128 = 10000000

+127 = 01111111
\end{homeworkProblem}

\begin{homeworkProblem}
Largest positive number represented in 32-bit normalized form: (2-2^-^2^3)*2^1^2^7

Smallest positive number represented in 32-bit normalized form: 2^-^1^4^9
\end{homeworkProblem}

\begin{homeworkProblem}
Largest negative number represented in 32-bit normalized form: -2^-^1^4^9

Smallest negative number represented in 32-bit normalized form: -(2-2^-^2^3)*2^1^2^7
\end{homeworkProblem}
\begin{homeworkProblem}
Largest positive number represented in 32-bit de-normalized form: (1-2^-^2^3)*2^-^1^2^6

Smallest positive number represented in 32-bit de-normalized form: 2^-^1^4^9
\end{homeworkProblem}
\begin{homeworkProblem}
Largest negative number represented in 32-bit de-normalized form: -2^-^1^4^9

Smallest negative number represented in 32-bit de-normalized form: -(1-2^-^2^3)*2^-^1^2^6
\end{homeworkProblem}
\begin{homeworkProblem}
Largest positive number represented in 64-bit normalized form: 2^1^0^2^3*(1+(1-2^-^5^2))

Smallest positive number represented in 64-bit normalized form: 2^-^1^0^2^2
\end{homeworkProblem}
\begin{homeworkProblem}
Largest negative number represented in 64-bit normalized form: -2^-^1^0^2^2

Smallest negative number represented in 64-bit normalized form: -2^1^0^2^3*(1+(1-2^-^5^2))
\end{homeworkProblem}
\begin{homeworkProblem}
Largest positive number represented in 64-bit de-normalized form: 2^-^1^0^2^2*(1-2^-^5^2)

Smallest positive number represented in 64-bit de-normalized form: 2^-^1^0^7^4
\end{homeworkProblem}
\begin{homeworkProblem}
Largest negative number represented in 64-bit de-normalized form: -2^-^1^0^7^4

Smallest negative number represented in 64-bit de-normalized form: -2^-^1^0^2^2*(1-2^-^5^2)
\end{homeworkProblem}
\end{document}   


